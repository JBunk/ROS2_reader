\chapter{Frequently Asked Questions}

\section{ROS?!}

\subsection{ondanks dat ik een package bouw blijft hij de oude gecompileerde bestanden uitvoeren?}
Waarschijnlijk bouw en run je op twee verschillende plekken. Waarschijnlijk ben je aan het bouwen in de package in plaats van in de hoofdfolder van je workspace.

\subsection{Ik heb al heel veel geprobeerd, maar de fout blijft maar bestaan?}
Waarschijnlijk bouw en run je op twee verschillende plekken. Waarschijnlijk ben je aan het bouwen in de package in plaats van in de hoofdfolder van je workspace.



\subsection{Rare errors}
\textit{Ik krijg rare errors?}

Waarschijnlijk ben je vergeten de ROS 2 environment te activeren door middel van:
\begin{lstlisting}[style=DOS, caption=Hello world, firstnumber=0, label={code:DOS_voorbeeld}]
 source /opt/ros/foxy/setup.bash
\end{lstlisting}
Dit moet je doen in elke terminal voordat je iets met ROS 2 doet (zowel runnen van nodes als het bouwen van packages).

\subsection{Colcon build geeft warning}
\textit{WARNING:colcon.colcon\_ros.prefix\_path.ament:The path '/een/path/' in the environment variable AMENT\_PREFIX\_PATH doesn't exist}
Waarschijnlijk heb je een package een andere naam gegeven. Deze warning kan geen kwaad. Wil je van de warning af? Verwijder dan de folders \textit{install}, \textit{build} en \textit{log} (rm -r build log install) en start een nieuwe terminal\footnote{Dit kan waarschijnlijk ook door AMENT\_PREFIX\_PATH te resetten. Laat even weten als je het command hiervoor hebt.}.

\subsection{Colcon build geeft error  message/service/action van andere package}
\begin{enumerate}
\item \textit{CMake Error at CMakeLists.txt:24 (find\_package):
  By not providing "Findpackage\_name" in CMAKE\_MODULE\_PATH this
  project has asked CMake to find a package configuration file provided by
  "packagename", but CMake did not find one.}

Je bent vergeten in je package.xml de package met de message/service/action toe te voegen aan aan het rijtje met <depend> </depend>.

\item \textit{CMake Error at /opt/ros/foxy/share/amen\_cmake\_target\_dependencies/cmake/ament\_target\_dependencies.cmake:77 (message):
  ament\_target\_dependencies() the passed package name `package\_name'}

Je bent vergeten de package met de message/action/service toe te voegen als dependencie aan je CMakeLists.txt

\end{enumerate}
\subsection{Message/service/action met een string}
\textit{Een string die ik via een message/service/action heb verstuurd print als rare tekens?}

Waarschijnlijk ben je vergeten om de functie \textit{c\_str()} te gebruiken. Je drukt nu de pointer af. Voorbeeld: Als je string in \textit{data} zit, dan krijg je de string met \textit{data.c\_str()}.

\subsection{Custom message/service/action hpp includeren}
\textit{Als ik een custom message/service/action heb gemaakt hoe heet dan de hpp die ik moet includeren?}
Een custom message/service/action wordt door ROS omgezet naar een .hpp die je moet includeren in de C++ bestanden die gebruik maken van de message/service/action. Het door ROS gemaakte .hpp-bestand hanteert een andere bestandsnaamconventie.
De naam van een custom message/service/action moet beginnen met een hoofdletter. Het hpp-bestand dat je moet includeren heeft vervolgens geen hoofdletters, maar voor elke hoofdletter, met uitzondering van de eerste, wordt een underscore (\_) geplaatst. Het pad van het hpp-bestand is gelijk aan dat van je custom message/action/service. In de codevoorbeelden die komen bij deze reader op de git-omgeving kan je hier voorbeelden van vinden.

\subsection{Error rond $>$ en ::SharedPointer bij het maken van een action Node}
Waarschijnlijk ben je rclcpp\_action vergeten toe te voegen aan je package.xml.

\subsection{Vtable}
\textit{undefined reference to `vtable for \dots'.}
Waarschijnlijk heeft je class/node nog geen destructor.

\section{Arduino?!}

\subsection{Serial Communicatie traag }
\textit{Als ik communiceer met serial op arduino kost dat veel tijd.}

Haal niet steeds het bericht op uit de buffer. Kijk eerst of er iets in de buffer staat door gebruik te maken van de functie \textit{Serial.available()}: \url{https://www.arduino.cc/reference/en/language/functions/communication/serial/available/}

